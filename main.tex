\documentclass[12pt,a4paper,english]{article}
\usepackage{tikz-cd}
\usepackage[a4paper]{geometry}
\usepackage{ctex}
\usepackage[utf8]{inputenc}
\usepackage[OT2,T1]{fontenc}
\usepackage{babel}
\usepackage{dsfont}
\usepackage{amsmath}
\usepackage{amssymb}
\usepackage{amsthm}
\usepackage{stmaryrd}
\usepackage{color}
\usepackage{array}
\usepackage{hyperref}
\usepackage{graphicx}
\usepackage{mathtools}
\usepackage{natbib}

\geometry{top=3cm,bottom=3cm,left=2.5cm,right=2.5cm}
\setlength\parindent{0pt}
\renewcommand{\baselinestretch}{1.3}

\newcommand\restr[2]{{% we make the whole thing an ordinary symbol
  \left.\kern-\nulldelimiterspace % automatically resize the bar with \right
  #1 % the function
  \vphantom{\big|} % pretend it's a little taller at normal size
  \right|_{#2} % this is the delimiter
  }}
  
% definition of the "structure"
\theoremstyle{plain}
\newtheorem{thm}{Theorem}[section]
\newtheorem{lem}[thm]{Lemma}
\newtheorem{prop}[thm]{Proposition}
\newtheorem{coro}[thm]{Corollary}
\theoremstyle{definition}
\newtheorem{defi}[thm]{Definition}
\newtheorem*{ex}{Example}
\newtheorem*{rem}{Remark}
\newtheorem{cla}[thm]{Claim}

\title{Hecke operators}
\date{December 3, 2021}
\author{Milan Berger-Guesneau, Deng Zhiyuan}


\begin{document}

\maketitle

\vspace{0.5cm}

The goal of this talk is to introduce Hecke operators and to describe their main properties.

We denote by $M_k$ (resp. $S_k$) the space of modular forms (resp. of cusp forms) for $SL_2(\mathbb{Z})$ of weight $k$. The discriminant modular form will be denoted by $\Delta$. Finally, we denote by $E_k$ (resp. $G_k$) the normalisation of the Eisenstein series $\sum c_n q^n$ of weight $k$ such that $c_0=1$ (resp. $c_1=1$) for $k\geqslant 4$.

Our main reference here is \cite[p.37-41]{zag}.
\vspace{0.5cm}

\textcolor{blue}{The text in blue has been omitted during the talk due to lack of time.}


\section{Definition and first properties}

\begin{defi}
Let $\mathcal{R}$ be the set\footnote{Actually $\mathcal{R}$ has a natural structure of complex manifold.} of all latices of $\mathbb{C}$. We say that a map $\mathcal{R}\to\mathbb{C}$ is homogeneous of degree $l$ if $F(\lambda L)=\lambda^l F(L)$ for every lattice $L$.
\end{defi}

There is a canonical bijection\footnote{See \cite[p.17]{RS}.}
\begin{align*}
   &\{F:\mathcal{R}\to\mathbb{C}\text{ holomorphic and homogeneous of degree }-k\}\to M_k\\
   & F\longmapsto (f:\tau\longmapsto F(\mathbb{Z}\oplus\tau\mathbb{Z}))
\end{align*}

We have seen last time that we can see modular forms as homogeneous functions of degree $-k$ from $\mathcal{R}$ to $\mathbb{C}$. 

\begin{defi}
Let $F:\mathcal{R}\to\mathbb{C}$ be homogeneous of degree $-k$. We define, for $m\geqslant 1$,
\begin{equation*}
    T_m(F)L=m^{k-1}\sum_{[L:L']=m}F\left(L'\right)
\end{equation*}
$T_m(F):\mathcal{R}\to\mathbb{C}$ is again homogeneous of degree $-k$.
\end{defi}


\begin{rem}
Because we can identify modular forms as homogeneous functions of degree $-k$ from $\mathcal{R}$ to $\mathbb{C}$, one obtain a operator
\begin{equation*}
    T_m:M_k\longrightarrow M_k
\end{equation*}
It's called the $m$th Hecke operator.

One can show that if $f=\sum_{n\geqslant 0}c_n q^n\in M_k$, then $T_m(f)=\sum_{n\geqslant 0}c_n' q^n$ where
\begin{equation}\label{defHecke}
    c_n':=\sum_{\substack{d\vert m,n\\
                  d>0}}
        d^{k-1}c_{mn/d^2}
\end{equation}
\end{rem}

\begin{rem}
From \eqref{defHecke} we get
\begin{equation*}
    c_0'=\sum_{d\mid m}d^{k-1}c_0=\sigma_{k-1}(m)c_0
\end{equation*}
and 
\begin{equation*}
    c_1'=c_m
\end{equation*}
Here we used the notation $\sigma_l(n)=\sum_{d\mid n}d^l$. In particular we have $c_0=0$ if and only if $c_0'=0$. It shows that the Hecke operators restricts to an operator
\begin{equation*}
    T_m:S_k\longrightarrow S_k
\end{equation*}
\end{rem}



\begin{thm}\label{dimMk}
\textcolor{blue}{
Let $k\in\mathbb{Z}$.
\begin{itemize}
    \item[(i)] The vector space $M_k$ is finite dimensional. Moreover, we have
\begin{align}\label{EqdimMk}
\dim M_k=\left\{
    \begin{array}{ll}
        0\text{ if }k<0\text{ or if }k\text{ is odd}\\
        \lfloor\frac{k}{12}\rfloor +1\text{ if }k\geqslant 0\text{ is even and }k\not\equiv 2\, [12]\\
        \lfloor\frac{k}{12}\rfloor \text{ if }k\geqslant 0\text{ is even and }k\equiv 2\, [12]
    \end{array}
\right.
\end{align}
\item[(ii)] If $\dim M_k\geqslant 1$, then the codimension of $S_k$ in $M_k$ is equal to $1$.
\end{itemize}
}
\end{thm}

\begin{proof}
\textcolor{blue}{
\textit{(i).} If $k$ is odd, then the modularity condition of $f\in M_k$ gives for $-I_2$ the relation $f=(-1)^kf=-f$ and so $f=0$.
Suppose now that $k$ is even.
}

\textcolor{blue}{
Let's separate the proof into three lemmas.
}
\begin{lem}
\textcolor{blue}{
If $k<0$, then $M_k=\{0\}$.
}
\end{lem}
\begin{proof}
\textcolor{blue}{
Let $f=\sum_{n\geqslant 0}c_n q^n\in M_k$ with $k<0$. Recall that if $\begin{pmatrix} a & b \\ c & d \end{pmatrix}\in SL_2(\mathbb{Z})$, then
\begin{align*}
\text{Im}\left(\frac{az+b}{cz+d}\right)=\frac{\text{Im}(z)}{|cz+d|^2}
\end{align*}
The modularity condition of $f$ implies
\begin{align*}
\left| f\left(\frac{az+b}{cz+d}\right)\right| \text{Im}\left(\frac{az+b}{cz+d}\right)^{k/2}&=|cz+d|^k\, |f(z)|\,\frac{\text{Im} (z)^{k/2}}{|cz+d|^k}\\
&=|f(z)|\,\text{Im} (z)^{k/2}
\end{align*}
The function $z\mapsto |f(z)|\,\text{Im} (z)^{k/2}$ is thus invariant under the action of $SL_2(\mathbb{Z})$. So we only have to study its values on the fundamental domain
\begin{align*}
\mathcal{D}=\left\{z\in\mathcal{H}\,\middle|\,|\text{Re}(z)|\leqslant\frac{1}{2}, |z|\geqslant 1\right\}
\end{align*}
We know that $|f(z)|\,\text{Im}(z)^{k/2}\to 0$ when $\text{Im}(z)\to +\infty$. Therefore there exists $\alpha>0$ such that if $\text{Im}(z)>\alpha$ then $|f(z)|\,\text{Im}(z)^{k/2}\leqslant 1$. Moreover this same function if bounded on the compact $\{z\in\mathcal{D}\mid \text{Im}(z)\leqslant\alpha\}$. It follows that there exists $C>0$ such that for all $z\in\mathcal{D}$ (and so for all $z\in\mathcal{H}$),
\begin{align*}
|f(z)|\,\text{Im}(z)^{k/2}\leqslant C
\end{align*}
Let $y>0$, let $m\in\mathbb{N}$. We have
\begin{align*}
f(x+iy)=\sum_{n\geqslant 0}c_n q^n=\sum_{n\geqslant 0}c_ne^{-2\pi ny}e^{2i\pi nx}
\end{align*}
and so
\begin{align*}
\int_0^1 f(x+iy)\,e^{-2i\pi mx}\,dx&=\sum_{nn\geqslant 0}c_n e^{-2\pi ny}\underbrace{\int_0^1  e^{2i\pi (n-m)x}\,dx}_{\delta_{m,n}}\\
&=c_m e^{-2\pi my}
\end{align*}
i.e.
\begin{align*}
c_m=e^{2\pi my}\int_0^1 f(x+iy)\,e^{-2i\pi mx}\,dx
\end{align*}
We can deduce
\begin{align}\label{esticm}
|c_m|\leqslant e^{2\pi my}\int_0^1 Cy^{-k/2}\,dx=C e^{2\pi my}y^{-k/2}
\end{align}
Because this quantity goes to $0$ when $y$ goes to $0$, we have $c_m=0$. We showed that all Fourier coefficient of $f$ are zero, i.e. $f=0$.
}
\end{proof}
\vspace{0.5cm}


\begin{lem}
\textcolor{blue}{
Part \textit{(i)} of the theorem is true for $k\in\{0,2,4,6,8,10\}$.
}
\end{lem}

\begin{proof}
\textcolor{blue}{
We start by treating the cases $k\in\{4,6,8,10\}$. Let $f\in M_k$. We remark that $f-c_0 E_k\in M_k$ and that its constant coefficient is zero. The function $(f-c_0 E_k)/\Delta$ is holomorphic on $\mathcal{H}$, and stays bounded when $\text{Im}(z)\to +\infty$. It also verifies the modularity conditions for $k-12$; it is therefore a modular form of weight $k-12<0$. By the preceding lemma, $f=c_0 E_k$ and so $M_k=\mathbb{C}E_k$.
}

\textcolor{blue}{
The case $k=0$ is very similar to the previous one, just replace $E_k$ by $1$. One then show $M_0=\mathbb{C}$.
}

\textcolor{blue}{
It remains to be seen the case $k=2$. Let $f\in M_2$. The modularity condition gives
\begin{align*}
f(i)=f\left(-\frac{1}{i}\right)=i^2 f(i)=-f(i)
\end{align*}
so $f(i)=0$. But $f^2\in M_4=\mathbb{C}E_4$; therefore there exists $\lambda\in\mathbb{C}$ such that $f^2=\lambda E_4$. Let's evaluate this equality for $i$: we obtain
\begin{align*}
0&=\lambda E_4(i)\\
&=\lambda\left(1-\frac{8}{B_4}\sum_{n\geqslant 0}\sigma_3(n)e^{2i\pi ni}\right)\\
&=\lambda\underbrace{\left(1+240\sum_{n\geqslant 0}\sigma_3(n)e^{-2\pi n}\right)}_{>0}
\end{align*}
It follows $\lambda=0$ and so $f=0$.
}
\end{proof}
\vspace{0.5cm}

\begin{lem}
\textcolor{blue}{
Let $k\geqslant 0$ be even. We have an isomorphism $\mathbb{C}\oplus M_{k}\cong M_{k+12}$.
}
\end{lem}
\begin{proof}
\textcolor{blue}{
Let $f\in M_k$. We know $(f-c_0 E_k)/\Delta\in M_{k}$; we can therefore write
\begin{align*}
f=c_0 E_k+g\Delta
\end{align*}
with $g\in M_{k}$. The linear map
\begin{align*}
\Phi:& \mathbb{C}\oplus M_{k}\to M_{k+12} \\
&(\lambda,g) \mapsto \lambda E_k+g\Delta
\end{align*}
is therefore surjective. It remains to show that $\Phi$ is injective. Suppose $\lambda E_k+g\Delta=0$. By looking at the constant coefficients we get $\lambda=0$. We therefore have $\Delta g=0$ and thus $g=0$. 
}
\end{proof}
\vspace{0.5cm}

\textcolor{blue}{
We can know finish the proof of \textit{(i)}: we showed $\dim M_{k+12}=\dim M_{k}+1$ if $k\geqslant 0$. Moreover, the right hand side of \eqref{EqdimMk} verifies the same inductive formula and coincides with $\dim M_k$ for $0\leqslant k<12$. Hence the result.
}
\vspace{1cm}

\textcolor{blue}{
\textit{(ii).} The identity $\dim M_k=\dim S_k+1$ is true for $k< 12$ by \textit{(i)}. It stays true for $k\geqslant 12$ because $\Phi$ restricts to an isomorphism $\mathbb{C}\oplus S_{k}\to S_{k+12}$.
}
\end{proof}


\begin{ex}
By theorem \ref{dimMk}, we have $\dim S_{12}=1$ and so $\{\Delta\}$ is a basis of $S_{12}$. It implies that $T_m\Delta$ is proportional to $\Delta$. We have
\begin{equation*}
    \Delta=q+\sum_{n\geqslant 2}\tau(n)q^n
\end{equation*}
and 
\begin{equation*}
    T_m\Delta=\tau(m)q+\sum_{n\geqslant 2}\tau(n)'q^n
\end{equation*}
Thus\footnote{We can see $\Delta$ as an eigenvector of Hecke operators.} $T_m\Delta=\tau(m)\Delta$. Equalizing the coefficients, we get
\begin{equation*}
    \sum_{\substack{d\vert m,n\\
                  d>0}}
        d^{11}\tau\left(\frac{mn}{d^2}\right)=\tau(m)\tau(n)
\end{equation*}
Thus we proved the
\begin{prop}
The function $\tau$ is multiplicative:
if $m$ and $n$ are coprime, then
\begin{equation*}
    \tau(mn)=\tau(m)\tau(n)
\end{equation*}
\end{prop}
\end{ex}



One can use the formula \eqref{defHecke} to prove the

\begin{prop} 
The Hecke operators verify the following properties:
\begin{itemize}
    \item[i)] If $(m,n)=1$, then
\begin{equation*}
    T_{mn}=T_{m}\circ T_{n}
\end{equation*}
    \item[ii)] If $l\geqslant 0$ and $p$ is prime, then 
\begin{equation*}
    T_{p^{l+2}}=T_{p^{l+1}}\circ T_{p}-p^{k-1}T_{p^{l}}
\end{equation*}
\end{itemize}
\end{prop}

It implies easily by induction the following result.

\begin{coro}\label{Tmcommute}
The Hecke operators commute with each other: for all $m,n\geqslant 1$,
\begin{equation*}
    T_m\circ T_n=T_n\circ T_m
\end{equation*}
\end{coro}


\section{Eigenforms of Hecke operators}

\begin{defi}
We say that a modular form $f=\sum c_n q^n$ is an eigenform if it is an eigenvector of all Hecke operators $T_m$, i.e. if there exists a sequence of complex numbers $(\lambda_m)$ such that $T_m f=\lambda_m f$ for all $m\geqslant 1$.

An eigenform $f$ is called normalised if $c_1=1$.
\end{defi}

\begin{ex}
We proved that $\Delta$ is a normalised eigenform. With similar arguments, one can show that the normalised Eisenstein series $G_k$ are also normalised eigenforms.
\end{ex}

\begin{prop}\label{CoefEigen}
If $f=\sum c_n q^n$ is a normalized eigenform, then
\begin{equation*}
    T_mf=c_m f
\end{equation*}
and
\begin{equation}\label{EqCoefEigen}
    \sum_{\substack{d\vert m,n\\
                  d>0}}
        d^{k-1}c_{mn/d^2}=c_mc_n
\end{equation}
In particular, the sequence $(c_n)$ is multiplicative.
\end{prop}
\begin{proof}
We already proved it for $\Delta$; the proof is the exact same in the general case.
\end{proof}

\begin{thm}
$S_k$ has a unique basis of normalised simultaneous eigenforms.
\end{thm}

\begin{proof}[Proof (sketch)]
One can show that the Hecke operators are diagonalizable. In order to prove this, one introduce an inner product\footnote{Called the Petersson inner product.} on the space of cusp forms. Let $\mathcal{H}$ be the upper complex plane. We define, for $f,g\in S_k$,
\begin{equation*}
    \langle f,g\rangle:=\int_{\mathcal{H}/SL_2(\mathbb{Z})}f(x+iy)\overline{g(x+iy)}y^k\frac{dx\,dy}{y^2}
\end{equation*}
One can show that this defines an inner product on $S_k$, and that the  Hecke operators are self-adjoint with respect to this inner product. This implies they are diagonalizable.
\vspace{0.5cm}

Because the Hecke operators commute with each other, we can apply the following result (proved in \cite{konrad}).
\begin{lem}
Let ${T_\alpha}$ be a set of commuting operators on a finite-dimensional
complex vector space $V$. If each $T_\alpha$ is diagonalizable on $V$ then they are simultaneously diagonalizable.
\end{lem}

\end{proof}

\begin{coro}
The Hecke operators are isomorphisms.
\end{coro}
\begin{proof}
Let $(f_i)$ be the basis of the preceding theorem. The operator $T_m$ sends this basis to the basis $(c_m^i f_i)$, and thus is an isomorphism.
\end{proof}

Let's now talk about $L$-functions associated to modular forms. We will see that they are pretty easy to understand in the case of eigenforms.

\begin{defi}
The $L$-function associated to a modular form $f=\sum c_n q^n$ is defined as the Dirichlet series
\begin{equation*}
    L(f,s):=\sum_{n\geqslant 1}\frac{c_n}{n^s}
\end{equation*}
\end{defi}

\begin{prop}
$L(f,s)$ absolutely converges on the half-plane $\text{\normalfont Re}\,s>1+k/2$.
\end{prop}

\begin{lem}
$c_n=O(n^{k/2})$
\end{lem}
\begin{proof}
\textcolor{blue}{Just take $y=1/n$ in equation \eqref{esticm}.}
\end{proof}


\begin{proof}[Proof of the proposition]
If $\text{\normalfont Re}\,s=1+k/2+\varepsilon$, then
\begin{equation*}
    \left|\frac{c_n}{n^s}\right|\leqslant\alpha n^{k/2-\text{\normalfont Re}\,s}=\alpha n^{-(1+\varepsilon)}
\end{equation*}
\end{proof}

\begin{thm}
Let $f=\sum c_n q^n\in M_k$ be a normalised eigenform. Then
\begin{equation}\label{eqL}
    L(f,s)=\prod_p\frac{1}{1-c_pp^{-s}+p^{k-1-2s}}
\end{equation}
\end{thm}

\begin{proof}
Let $S_p:=\sum_{l\geqslant 0}c_{p^l}p^{-ls}$. Using the fact that $(c_n)$ is multiplicative (by proposition \eqref{CoefEigen}), we have
\begin{align*}
    \prod_p S_p&=\prod_p\left(\sum_{l\geqslant 0}\frac{c_{p^l}}{p^{ls}}\right)\\
    &=\sum_{p_1^{\alpha_1}\cdots p_k^{\alpha_k}}\frac{c_{p_1^{\alpha_1}}\ldots c_{p_k^{\alpha_k}}}{p_1^{\alpha_1s}\ldots p_k^{\alpha_ks}}\\
    &=\sum_{n\geqslant 1}\frac{c_n}{n^s}\\
    &=L(f,s)
\end{align*}
On the other hand, the identity \eqref{EqCoefEigen} implies for $m=p^{l+1}$ and $n=p$ the relation
\begin{equation*}
    c_{p^{l+2}}=c_{p^{l+1}}c_p-p^{k-1}c_{p^l}
\end{equation*}
Thus
\begin{align*}
    S_p&=1+\frac{c_p}{p^s}+\sum_{l\geqslant 0}c_{p^{l+2}}p^{-(l+2)s}\\
    &=1+\frac{c_p}{p^s}+\left(\frac{c_p}{p^s}-\frac{p^{k-1}}{p^{2s}}\right)S_p-\frac{c_p}{p^s}\\
    &=1+\left(\frac{c_p}{p^s}-p^{k-1-2s}\right)S_p
\end{align*}
and so
\begin{equation*}
    S_p=\frac{1}{1-c_p p^{-s}-p^{k-1-2s}}
\end{equation*}
\end{proof}

One can show that the converse of this theorem is also true: if the $L$-function of $f$ is of the form \eqref{eqL}, then $f$ is an eigenform.

\begin{ex}
We know that the normalised Eisenstein series\footnote{Here $B_k$ is the $k$th Bernoulli number.} $G_k=-B_k/2k+\sum_{n\geqslant 1}\sigma_{k-1}(n)q^n$ are eigenforms. In this case,
\begin{equation*}
    c_p=\sigma_{k-1}(p)=\sum_{d\mid p}d^{k-1}=1+p^{k-1}
\end{equation*}
and so
\begin{align*}
    L(G_k,s)&=\prod_p\frac{1}{1-(1+p^{k-1})p^{-s}+p^{k-1-2s}}\\
    &=\left(\prod_p\frac{1}{1+p^{-s}}\right)\left(\prod_p\frac{1}{1+p^{-(s-k+1)}}\right)\\
    &=\zeta(s)\zeta(s-k+1)
\end{align*}
\textcolor{blue}{
Actually there is an other way to calculate $L(G_k,s)$. Indeed, we have the equality of arithmetic functions $\sigma_{k-1}=h_{k-1} * \mathbf{1}$ where $h_{k-1}(n)=n^{k-1}$, $\mathbf{1}(n)=1$ and $*$ is the convolution product given by
\begin{equation*}
    (f*g)(n)=\sum_{d\mid n}f(d)g\left(\frac{n}{d}\right)
\end{equation*}
It follows that
\begin{equation*}
    L(G_k,s)=L(\mathbf{1},s)L(h_{k-1},s)=\zeta(s)\zeta(s-k+1)
\end{equation*}
}
\end{ex}
where $\zeta$ is the Riemann zeta function.

Those $L$-functions are very important for several reasons, one of them being that they provide another link between modular forms and elliptic curves. For instance, we have the following result. This is exactly the kind of results that Langlands program tries to generalise.

\begin{thm}[Eichler–Shimura]
Let $f=\sum c_n q^n\in S_2$ be an eigenform with $c_n\in\mathbb{Z}$ for all $n$. Then there exists an elliptic curve $E$ over $\mathbb{Q}$ such that $L(f, s)=L(E/\mathbb{Q}, s)$.
\end{thm}

We won't give the definition of the definition of the $L$-function associated to an elliptic curve, see \cite[p.44]{zag} for more details.

We conclude this talk with a very deep theorem, which has been used to prove the last Fermat theorem. For now we only defined modular forms for $SL_2(\mathbb{Z})$, but one can also consider modular forms over subgroups of $SL_2(\mathbb{Z})$. Here is an example:

\begin{align*}
\Gamma_0(N)=\left\lbrace\begin{pmatrix} a & b \\ c & d \end{pmatrix}\in SL_2(\mathbb{Z})\,\middle|\, N\mid c\right\rbrace
\end{align*}
It's a finite index subgroup of $SL_2(\mathbb{Z})$.

\begin{thm}[Modularity theorem]
Let $E$ be an elliptic curve over $\mathbb{Q}$. There exists an $N\geqslant 1$ and a cusp form $f$ of weight $2$ on $\Gamma_0(N)$ such that $L(E/\mathbb{Q}, s) = L(f, s)$.
\end{thm}

\newpage
\bibliographystyle{unsrt}
\bibliography{bib.bib}
\end{document}
